\section{Introdução}

\hspace*{3em} Para essa análise nos restringiremos a analisar apenas o UNO com uma pessoa, que iremos modelar a partir de um grafo bipartido. Para facilitar o entendimento, iremos dar uma visão geral do passo a passo da prova da NP-completude do UNO-1. Partiremos de um grafo bipartido (cor e número), onde as arestas representam as cartas. Transformaremos esse grafo em um grafo de linha (assim, as arestas passam a ser representadas por vértices).

\hspace*{3em} No grafo de linha, os vértices correspondem às cartas do jogo, e duas cartas serão conectadas por uma aresta caso sejam jogáveis consecutivamente de acordo com as regras do UNO (mesma cor ou mesmo número). Em seguida, mostraremos como a resolução do problema UNO-1 está relacionada a encontrar um caminho hamiltoniano nesse grafo de linha.

\hspace*{3em} A partir daí, provaremos que o problema de decidir se existe uma sequência válida de cartas para vencer o UNO-1 é NP-completo, utilizando uma redução de um problema clássico já conhecido como NP-completo.

\hspace*{2em} Para isso iremos primeiro definir uma notação matemática para o jogo, que permitirá essa modelagem.

\subsection{Notação}
    \subsubsection*{Cartas}
        \hspace*{3em} No jogo cada carta é representada por um par ordenado $(c, n)$, onde $c$ é a cor da carta e $n$ é o número da carta. Onde $c \in \{0, 1, 2, 3\}$, onde $0$ é a cor vermelha, $1$ é a cor amarela, $2$ é a cor verde e $3$ é a cor azul, e $n \in \{0, 1, 2, 3, 4, 5, 6, 7, 8\}$, tal que $0$ é a carta 1, $1$ é a carta 2, $2$ é a carta 3, ... e $8$ e 9.

    \subsubsection*{Jogada}
        \hspace*{3em} Em uma determinada jogada, o jogador pode jogar uma carta que seja igual à carta que está na mesa, ou seja, que tenha a mesma cor ou o mesmo tipo. Caso contrário, ele pode comprar uma carta do monte e, se a carta comprada puder ser jogada, ele pode jogá-la.

    \subsubsection*{Descarte}
        \hspace*{3em} Quando um jogador descarta uma carta, ele deve colocá-la no topo da pilha de descarte. A carta no topo da pilha de descarte determina as regras para a próxima jogada. Se a pilha de descarte estiver vazia, qualquer carta pode ser jogada para iniciar a pilha.

        \hspace*{2em} Em outras palavras, uma carta $x_i = (c, n)$ pode ser jogada se a carta $x_{i-1} = (c', n')$ no topo da pilha de descarte satisfizer uma das seguintes condições:
        \begin{itemize}
            \item $(c = c') \lor  (n = n')$
        \end{itemize}

        \subsubsection*{Regras}
            \hspace*{3em} Vamos definir as regras do jogo usando a seguinte notação matemática:
            
            \begin{itemize}
                \item Seja $C$ o conjunto de todas as cartas, onde $C = \{(c, n) \mid c \in \{0, 1, 2, 3\}, n \in \{0, 1, 2, 3, 4, 5, 6, 7, 8\}\}$.
                \item Seja $D$ a pilha de descarte, onde $D = [d_1, d_2, \ldots, d_n]$ e $d_i \in C$ para todo $i$.
                \item Seja $M$ o monte de compra, onde $M = [m_1, m_2, \ldots, m_k]$ e $m_i \in C$ para todo $i$.
                \item Seja $P$ o conjunto de jogadores, onde $P = \{p_1, p_2, \ldots, p_j\}$.
                \item Seja $H(p)$ a mão de cartas do jogador $p$, onde $H(p) \subseteq C$.
            \end{itemize}

            \hspace*{3em} As regras do jogo podem ser descritas da seguinte forma:
            
            \begin{enumerate}
                        \item Um jogador $p \in P$ pode jogar uma carta $c \in H(p)$ se $c = (c_1, t_1)$ e a carta no topo da pilha de descarte $d_n = (c_2, t_2)$, \textbf{se uma carta corresponde à outra}, em outra palabres, satisfaz uma das seguintes condições:
                        \begin{equation}
                            c_1 = c_2 \lor t_1 = t_2
                        \end{equation}
                \item Se o jogador não puder jogar uma carta, ele deve comprar uma carta do monte $M$. Se a carta comprada puder ser jogada, ele pode jogá-la imediatamente. 
                \item Quando um jogador joga uma carta, essa carta é adicionada ao topo da pilha de descarte $D$.
            \end{enumerate}


            
